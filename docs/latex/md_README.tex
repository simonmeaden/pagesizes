A List of supported types with their enum selection values.

Please note that this is by no means a totally complete list. It was all of those that I found with are fairly simple search of the internet. If you need to add to them either let me know on \href{mailto:simonmeaden@sky.com}{\tt simonmeaden@sky.\+com}, giving names and sizes or a link to a website containing same. Alternatively do it yourselfby forking the github site and uploading the changes.

\subsubsection*{I\+SO Sizes.}

\paragraph*{A Series}

The dimensions of the A series paper sizes are defined by the I\+SO 216 international paper size standard. The A series was adopted in Europe in the 19th century, and is currently used all around the world, apart from in the U\+SA and Canada. The most common paper size used in English speaking countries around the world is A4, which is 210mm x 297mm (8.\+27 inches x 11.\+7 inches). The largest sheet from the A series is the A0 size of paper. It has an area of 1m2, and the dimensions are 841mm x 1189mm. The A series uses an aspect ratio of 1\+:√2, and other sizes in the series are defined by folding the paper in half, parallel to its smaller sides. For example, cutting an A4 in half, will create an two A5 sheets, and so forth. Any size of brochure can be made using paper from the next larger size, for example A3 sheets are folded to make A4 brochures. The standard lengths and widths of the A series are rounded to the nearest millimetre.

\tabulinesep=1mm
\begin{longtabu} spread 0pt [c]{*{3}{|X[-1]}|}
\hline
\rowcolor{\tableheadbgcolor}\textbf{ enum  }&\multicolumn{2}{p{(\linewidth-\tabcolsep*3-\arrayrulewidth*2)*2/3}|}{\cellcolor{\tableheadbgcolor}\textbf{ Size Nam   }}\\\cline{1-3}
\endfirsthead
\hline
\endfoot
\hline
\rowcolor{\tableheadbgcolor}\textbf{ enum  }&\multicolumn{2}{p{(\linewidth-\tabcolsep*3-\arrayrulewidth*2)*2/3}|}{\cellcolor{\tableheadbgcolor}\textbf{ Size Nam   }}\\\cline{1-3}
\endhead
I\+S\+O\+\_\+4\+A0  &4\+A0  &2378 x 1682   \\\cline{1-3}
I\+S\+O\+\_\+2\+A0  &2\+A0  &1682 x 1189   \\\cline{1-3}
I\+S\+O\+\_\+\+A0  &A0  &1189 x 841   \\\cline{1-3}
I\+S\+O\+\_\+\+A0\+\_\+\+Plus  &A0+  &914 x 1292   \\\cline{1-3}
I\+S\+O\+\_\+\+A1  &A1  &841 x 594   \\\cline{1-3}
I\+S\+O\+\_\+\+A1\+\_\+\+Plus  &A1+  &609 x 914   \\\cline{1-3}
I\+S\+O\+\_\+\+A2  &A2  &594 x 420   \\\cline{1-3}
I\+S\+O\+\_\+\+A3  &A3  &420 x 297   \\\cline{1-3}
I\+S\+O\+\_\+\+A3\+\_\+\+Plus  &A3+  &329 x 483   \\\cline{1-3}
I\+S\+O\+\_\+\+A4  &A4  &297 x 210   \\\cline{1-3}
I\+S\+O\+\_\+\+A5  &A5  &210 x 148   \\\cline{1-3}
I\+S\+O\+\_\+\+A6  &A6  &148 x 105   \\\cline{1-3}
I\+S\+O\+\_\+\+A7  &A7  &105 x 74   \\\cline{1-3}
I\+S\+O\+\_\+\+A8  &A8  &74 x 52   \\\cline{1-3}
I\+S\+O\+\_\+\+A9  &A9  &52 x 37   \\\cline{1-3}
I\+S\+O\+\_\+\+A10  &A10  &37 x 26   \\\cline{1-3}
\end{longtabu}


\paragraph*{B Series}

The dimensions of the B series paper sizes are defined by the I\+SO 216 international paper size standard. The B series is not as common as the A series. It was created to provide paper sizes that were not included in the A series. The B series uses an aspect ratio of 1\+:√2. The area of the B series paper sheets is the geometric mean of the A series sheets. For example, B1 is between A0 and A1 in size. While the B series is less common in office use, it is more regularly used in other special situations, such as posters, books, envelopes and passports.

\tabulinesep=1mm
\begin{longtabu} spread 0pt [c]{*{3}{|X[-1]}|}
\hline
\rowcolor{\tableheadbgcolor}\textbf{ enum  }&\multicolumn{2}{p{(\linewidth-\tabcolsep*3-\arrayrulewidth*2)*2/3}|}{\cellcolor{\tableheadbgcolor}\textbf{ Size Nam   }}\\\cline{1-3}
\endfirsthead
\hline
\endfoot
\hline
\rowcolor{\tableheadbgcolor}\textbf{ enum  }&\multicolumn{2}{p{(\linewidth-\tabcolsep*3-\arrayrulewidth*2)*2/3}|}{\cellcolor{\tableheadbgcolor}\textbf{ Size Nam   }}\\\cline{1-3}
\endhead
I\+S\+O\+\_\+\+B0  &B0  &1000 × 1414   \\\cline{1-3}
I\+S\+O\+\_\+\+B1  &B1  &707 × 1000   \\\cline{1-3}
I\+S\+O\+\_\+\+B2  &B2  &500 × 707   \\\cline{1-3}
I\+S\+O\+\_\+\+B3  &B3  &353 × 500   \\\cline{1-3}
I\+S\+O\+\_\+\+B4  &B4  &250 × 353   \\\cline{1-3}
I\+S\+O\+\_\+\+B5  &B5  &176 × 250   \\\cline{1-3}
I\+S\+O\+\_\+\+B6  &B6  &125 × 176   \\\cline{1-3}
I\+S\+O\+\_\+\+B7  &B7  &88 × 125   \\\cline{1-3}
I\+S\+O\+\_\+\+B8  &B8  &62 × 88   \\\cline{1-3}
I\+S\+O\+\_\+\+B9  &B9  &44 × 62   \\\cline{1-3}
I\+S\+O\+\_\+\+B10  &B10  &31 × 44 mm   \\\cline{1-3}
\end{longtabu}


\paragraph*{C Series}

The dimensions of the C series sizes are defined by the I\+SO 269 paper size standard. The C series is most commonly used for envelopes. The area of C series paper is the geometric mean of the areas of the A and B series paper of the same number. For example, C4 has a bigger area than A4, but smaller area than B4. Therefore an A4 piece of paper will fit into a C4 envelope. The aspect ratio of C series envelopes is 1\+:√2, and this means that an A4 sheet of paper when folded in half, parallel to its smaller sides, will fit nicely into a C5 envelope. When it is folded twice, it will fit into a C6 envelope, and so on.

\tabulinesep=1mm
\begin{longtabu} spread 0pt [c]{*{3}{|X[-1]}|}
\hline
\rowcolor{\tableheadbgcolor}\textbf{ enum  }&\multicolumn{2}{p{(\linewidth-\tabcolsep*3-\arrayrulewidth*2)*2/3}|}{\cellcolor{\tableheadbgcolor}\textbf{ Size Nam   }}\\\cline{1-3}
\endfirsthead
\hline
\endfoot
\hline
\rowcolor{\tableheadbgcolor}\textbf{ enum  }&\multicolumn{2}{p{(\linewidth-\tabcolsep*3-\arrayrulewidth*2)*2/3}|}{\cellcolor{\tableheadbgcolor}\textbf{ Size Nam   }}\\\cline{1-3}
\endhead
I\+S\+O\+\_\+\+C0  &C0  &917 × 1297   \\\cline{1-3}
I\+S\+O\+\_\+\+C1  &C1  &648 × 917   \\\cline{1-3}
I\+S\+O\+\_\+\+C2  &C2  &458 × 648   \\\cline{1-3}
I\+S\+O\+\_\+\+C3  &C3  &324 × 458   \\\cline{1-3}
I\+S\+O\+\_\+\+C4  &C4  &229 × 324   \\\cline{1-3}
I\+S\+O\+\_\+\+C5  &C5  &162 × 229   \\\cline{1-3}
I\+S\+O\+\_\+\+C6  &C6  &114 × 162   \\\cline{1-3}
I\+S\+O\+\_\+\+C7  &C7  &81 × 114   \\\cline{1-3}
I\+S\+O\+\_\+\+C8  &C8  &57 × 81   \\\cline{1-3}
I\+S\+O\+\_\+\+C9  &C9  &40 × 57   \\\cline{1-3}
I\+S\+O\+\_\+\+C10  &C10  &28 × 40   \\\cline{1-3}
\end{longtabu}


\subsubsection*{Envelope Sizes}

\tabulinesep=1mm
\begin{longtabu} spread 0pt [c]{*{3}{|X[-1]}|}
\hline
\rowcolor{\tableheadbgcolor}\textbf{ num  }&\multicolumn{2}{p{(\linewidth-\tabcolsep*3-\arrayrulewidth*2)*2/3}|}{\cellcolor{\tableheadbgcolor}\textbf{ Size Name   }}\\\cline{1-3}
\endfirsthead
\hline
\endfoot
\hline
\rowcolor{\tableheadbgcolor}\textbf{ num  }&\multicolumn{2}{p{(\linewidth-\tabcolsep*3-\arrayrulewidth*2)*2/3}|}{\cellcolor{\tableheadbgcolor}\textbf{ Size Name   }}\\\cline{1-3}
\endhead
I\+S\+O\+\_\+\+Envelope\+\_\+\+DL  &DL  &110 x 220   \\\cline{1-3}
I\+S\+O\+\_\+\+Envelope\+\_\+\+B4  &B4  &353 x 250   \\\cline{1-3}
I\+S\+O\+\_\+\+Envelope\+\_\+\+B5  &B5  &250 x 176   \\\cline{1-3}
I\+S\+O\+\_\+\+Envelope\+\_\+\+B6  &B6  &176 x 125   \\\cline{1-3}
I\+S\+O\+\_\+\+Envelope\+\_\+\+C0  &C0  &917 x 1297   \\\cline{1-3}
I\+S\+O\+\_\+\+Envelope\+\_\+\+C1  &C1  &648 x 914   \\\cline{1-3}
I\+S\+O\+\_\+\+Envelope\+\_\+\+C2  &C2  &458 x 648   \\\cline{1-3}
I\+S\+O\+\_\+\+Envelope\+\_\+\+C3  &C3  &324 x 458   \\\cline{1-3}
I\+S\+O\+\_\+\+Envelope\+\_\+\+C4  &C4  &229 x 324   \\\cline{1-3}
I\+S\+O\+\_\+\+Envelope\+\_\+\+C5  &C5  &162 x 229   \\\cline{1-3}
I\+S\+O\+\_\+\+Envelope\+\_\+\+C6\+\_\+\+C5  &C6/\+C5  &229 x 114   \\\cline{1-3}
I\+S\+O\+\_\+\+Envelope\+\_\+\+C6  &C6  &114 x 162   \\\cline{1-3}
I\+S\+O\+\_\+\+Envelope\+\_\+\+C7\+\_\+\+C6  &C7/\+C6  &81 x 16   \\\cline{1-3}
I\+S\+O\+\_\+\+Envelope\+\_\+\+C7  &C7  &81 x 114   \\\cline{1-3}
I\+S\+O\+\_\+\+Envelope\+\_\+\+E4  &E4  &400 x 280   \\\cline{1-3}
\end{longtabu}


\paragraph*{Raw Untrimmed Sizes}

The dimensions of the R\+AW series paper sizes are defined by the I\+SO 217\+:1995 paper size standard. The format consists of the RA series and the S\+RA series. RA stands for ‘\+Raw Format A’, and S\+RA stands for ‘\+Supplementary Raw Format A’. The untrimmed raw paper is used for commercial printing, and as it is slightly larger than the A series format, it allows the ink to bleed to the edge of the paper, before being cut to match the A format.

enum $\vert$ Size Name $\vert$ Size in mm --- $\vert$ --- $\vert$ --- R\+A\+W\+\_\+\+R\+A0 $\vert$ R\+A0 $\vert$$\vert$860 × 1220 R\+A\+W\+\_\+\+R\+A1 $\vert$\+R\+A1 $\vert$610 × 860 R\+A\+W\+\_\+\+R\+A2 $\vert$\+R\+A2 $\vert$430 × 610 R\+A\+W\+\_\+\+R\+A3 $\vert$ R\+A3$\vert$305 × 430 R\+A\+W\+\_\+\+R\+A4 $\vert$ R\+A4$\vert$215 × 305 R\+A\+W\+\_\+\+S\+R\+A0 $\vert$\+S\+R\+A0$\vert$900 × 1280 R\+A\+W\+\_\+\+S\+R\+A1 $\vert$\+S\+R\+A1 $\vert$640 × 900 R\+A\+W\+\_\+\+S\+R\+A2 $\vert$ S\+R\+A2$\vert$450 × 640 R\+A\+W\+\_\+\+S\+R\+A3 $\vert$ S\+R\+A3$\vert$320 × 450 R\+A\+W\+\_\+\+S\+R\+A4 $\vert$\+S\+R\+A4 $\vert$225 × 320 R\+A\+W\+\_\+\+S\+R\+A1\+\_\+\+Plus $\vert$ S\+R\+A1$\vert$660 × 920 R\+A\+W\+\_\+\+S\+R\+A2\+\_\+\+Plus $\vert$ S\+R\+A2$\vert$480 × 650 R\+A\+W\+\_\+\+S\+R\+A3\+\_\+\+Plus $\vert$ S\+R\+A3$\vert$320 × 460 R\+A\+W\+\_\+\+S\+R\+A3\+\_\+\+Plus\+\_\+\+Plus $\vert$ S\+R\+A3$\vert$320 × 464 R\+A\+W\+\_\+\+A0U $\vert$ A0\+U$\vert$880 × 1230 R\+A\+W\+\_\+\+A1U $\vert$\+A1U $\vert$625 × 880 R\+A\+W\+\_\+\+A2U $\vert$\+A2U $\vert$450 × 625 R\+A\+W\+\_\+\+A3U $\vert$ A3\+U$\vert$330 × 450 R\+A\+W\+\_\+\+A4U $\vert$\+A4U $\vert$240 × 330

\paragraph*{Transitional Sizes}

The transitional paper size formats consist of the PA Series and the F Series. The PA Series proposed to be included into the I\+SO 216 standard in 1975. They were rejected by the committee who decided that the number of standardised paper formats should be kept to a minimum. However this series is still used today, in particular P\+A4 (or L4). Its width is that of the I\+SO standard A4, and height of Canadian P4 (210mm x 280mm). The P\+A4 format can easily be printed on equipment designed for A4 or US Letter size, which is why it is used by many international magazines, and is a good solution as the format of presentation slides. P\+A4 is more of a page format than a paper size. It has a 4\+:3 aspect ratio when in landscape orientation.

In Southeast Asia, a common size used is the non-\/standard F4. The longer side is 330mm, which is the same as the British Foolscap. The shorter side is 210mm, which is the same as the I\+SO A4. The F4 is occasionally known as (metric) Foolscap or Folio.

\tabulinesep=1mm
\begin{longtabu} spread 0pt [c]{*{3}{|X[-1]}|}
\hline
\rowcolor{\tableheadbgcolor}\textbf{ enum  }&\multicolumn{2}{p{(\linewidth-\tabcolsep*3-\arrayrulewidth*2)*2/3}|}{\cellcolor{\tableheadbgcolor}\textbf{ Size Nam   }}\\\cline{1-3}
\endfirsthead
\hline
\endfoot
\hline
\rowcolor{\tableheadbgcolor}\textbf{ enum  }&\multicolumn{2}{p{(\linewidth-\tabcolsep*3-\arrayrulewidth*2)*2/3}|}{\cellcolor{\tableheadbgcolor}\textbf{ Size Nam   }}\\\cline{1-3}
\endhead
Transitional\+\_\+\+P\+A0  &P\+A0  &840 × 1120   \\\cline{1-3}
Transitional\+\_\+\+P\+A1  &P\+A1  &560 × 840   \\\cline{1-3}
Transitional\+\_\+\+P\+A2  &P\+A2  &420 × 560   \\\cline{1-3}
Transitional\+\_\+\+P\+A3  &P\+A3  &280 × 420   \\\cline{1-3}
Transitional\+\_\+\+P\+A4  &P\+A4  &210 × 280   \\\cline{1-3}
Transitional\+\_\+\+P\+A5  &P\+A5  &140 × 210   \\\cline{1-3}
Transitional\+\_\+\+P\+A6  &P\+A6  &105 × 140   \\\cline{1-3}
Transitional\+\_\+\+P\+A7  &P\+A7  &70 × 105   \\\cline{1-3}
Transitional\+\_\+\+P\+A8  &P\+A8  &52 × 70   \\\cline{1-3}
Transitional\+\_\+\+P\+A9  &P\+A9  &35 × 52   \\\cline{1-3}
Transitional\+\_\+\+P\+A10  &P\+A10  &26 × 35   \\\cline{1-3}
Transitional\+\_\+\+F0  &F0  &841 × 1321   \\\cline{1-3}
Transitional\+\_\+\+F1  &F1  &660 × 841   \\\cline{1-3}
Transitional\+\_\+\+F2  &F2  &420 × 660   \\\cline{1-3}
Transitional\+\_\+\+F3  &F3  &330 × 420   \\\cline{1-3}
Transitional\+\_\+\+F4  &F4  &210 × 330   \\\cline{1-3}
Transitional\+\_\+\+F5  &F5  &165 × 210   \\\cline{1-3}
Transitional\+\_\+\+F6  &F6  &105 × 165   \\\cline{1-3}
Transitional\+\_\+\+F7  &F7  &82 × 105   \\\cline{1-3}
Transitional\+\_\+\+F8  &F8  &52 × 82   \\\cline{1-3}
Transitional\+\_\+\+F9  &F9  &41 × 52   \\\cline{1-3}
Transitional\+\_\+\+F10  &F10  &26 × 41   \\\cline{1-3}
\end{longtabu}


\subsubsection*{US Sizes}

\paragraph*{Loose Sizes}

Paper sizes in North America do not follow a logical system like the I\+SO standard does. They have their own system that they follow. This means that scaling the paper sizes is more difficult. The US Letter paper size is the most popular size used throughout the US. It is also commonly used in Chile and the Philippines.

\tabulinesep=1mm
\begin{longtabu} spread 0pt [c]{*{3}{|X[-1]}|}
\hline
\rowcolor{\tableheadbgcolor}\textbf{ enum  }&\multicolumn{2}{p{(\linewidth-\tabcolsep*3-\arrayrulewidth*2)*2/3}|}{\cellcolor{\tableheadbgcolor}\textbf{ Size Nam   }}\\\cline{1-3}
\endfirsthead
\hline
\endfoot
\hline
\rowcolor{\tableheadbgcolor}\textbf{ enum  }&\multicolumn{2}{p{(\linewidth-\tabcolsep*3-\arrayrulewidth*2)*2/3}|}{\cellcolor{\tableheadbgcolor}\textbf{ Size Nam   }}\\\cline{1-3}
\endhead
U\+S\+\_\+\+Half\+\_\+\+Letter  &Half letter  &140 × 216   \\\cline{1-3}
U\+S\+\_\+\+Letter  &Letter  &216 × 279   \\\cline{1-3}
U\+S\+\_\+\+Legal  &Legal  &216 × 356   \\\cline{1-3}
U\+S\+\_\+\+Junior\+\_\+\+Legal  &Junior Legal  &127 × 203   \\\cline{1-3}
U\+S\+\_\+\+Ledger  &Ledger  &432 × 279   \\\cline{1-3}
U\+S\+\_\+\+Tabloid  &Tabloid  &279 × 432   \\\cline{1-3}
U\+S\+\_\+\+Gov\+\_\+\+Letter  &Government Letter  &203 × 267   \\\cline{1-3}
U\+S\+\_\+\+Gov\+\_\+\+Legal  &Government Legal  &216 × 330   \\\cline{1-3}
\end{longtabu}


\paragraph*{A\+N\+SI}

The American National Standards Institute (A\+N\+SI) adopted A\+N\+S\+I/\+A\+S\+ME Y14.\+1, which defined a series of paper sizes in 1996, based on the standard 8.\+5 inches x 11 inches (216mm x 279mm) Letter size, which was named ‘\+A\+N\+SI A’. This series is fairly similar to the I\+SO standard, in that if you cut a sheet in half, you will product two sheets of the next smaller size. Ledger/\+Tabloid is known as ‘\+A\+N\+SI B’. The most common and widely used size is A\+N\+SI A, also known as ‘\+Letter’.

\tabulinesep=1mm
\begin{longtabu} spread 0pt [c]{*{3}{|X[-1]}|}
\hline
\rowcolor{\tableheadbgcolor}\textbf{ enum  }&\multicolumn{2}{p{(\linewidth-\tabcolsep*3-\arrayrulewidth*2)*2/3}|}{\cellcolor{\tableheadbgcolor}\textbf{ Size Nam   }}\\\cline{1-3}
\endfirsthead
\hline
\endfoot
\hline
\rowcolor{\tableheadbgcolor}\textbf{ enum  }&\multicolumn{2}{p{(\linewidth-\tabcolsep*3-\arrayrulewidth*2)*2/3}|}{\cellcolor{\tableheadbgcolor}\textbf{ Size Nam   }}\\\cline{1-3}
\endhead
U\+S\+\_\+\+A\+N\+S\+I\+\_\+A  &A\+N\+SI A  &216 × 279   \\\cline{1-3}
U\+S\+\_\+\+A\+N\+S\+I\+\_\+B  &A\+N\+SI B  &279 × 432   \\\cline{1-3}
U\+S\+\_\+\+A\+N\+S\+I\+\_\+C  &A\+N\+SI C  &432 × 559   \\\cline{1-3}
U\+S\+\_\+\+A\+N\+S\+I\+\_\+D  &A\+N\+SI D  &559 × 864   \\\cline{1-3}
U\+S\+\_\+\+A\+N\+S\+I\+\_\+E  &A\+N\+SI E  &864 × 1118   \\\cline{1-3}
\end{longtabu}


\paragraph*{Architectural Sizes}

The Architectural series (A\+R\+CH) is used by architects in North America, and they prefer to use this series instead of A\+N\+SI, because the aspect ratios are ratios of smaller whole numbers (4\+:3 and 3\+:2). The A\+R\+CH series of paper sizes is defined in the A\+N\+S\+I/\+A\+S\+ME Y14.\+1 standard. The A\+R\+CH sizes are commonly used by architects for their large format drawings.

\tabulinesep=1mm
\begin{longtabu} spread 0pt [c]{*{3}{|X[-1]}|}
\hline
\rowcolor{\tableheadbgcolor}\textbf{ enum  }&\multicolumn{2}{p{(\linewidth-\tabcolsep*3-\arrayrulewidth*2)*2/3}|}{\cellcolor{\tableheadbgcolor}\textbf{ Size Nam   }}\\\cline{1-3}
\endfirsthead
\hline
\endfoot
\hline
\rowcolor{\tableheadbgcolor}\textbf{ enum  }&\multicolumn{2}{p{(\linewidth-\tabcolsep*3-\arrayrulewidth*2)*2/3}|}{\cellcolor{\tableheadbgcolor}\textbf{ Size Nam   }}\\\cline{1-3}
\endhead
U\+S\+\_\+\+Arch\+\_\+A  &Arch A  &229 × 305   \\\cline{1-3}
U\+S\+\_\+\+Arch\+\_\+B  &Arch B  &305 × 457   \\\cline{1-3}
U\+S\+\_\+\+Arch\+\_\+C  &Arch C  &457 × 610   \\\cline{1-3}
U\+S\+\_\+\+Arch\+\_\+D  &Arch D  &610 × 914   \\\cline{1-3}
U\+S\+\_\+\+Arch\+\_\+E  &Arch E  &914 × 1219   \\\cline{1-3}
U\+S\+\_\+\+Arch\+\_\+\+E1  &Arch E1  &762 × 1067   \\\cline{1-3}
U\+S\+\_\+\+Arch\+\_\+\+E2  &Arch E2  &660 × 965   \\\cline{1-3}
U\+S\+\_\+\+Arch\+\_\+\+E3  &Arch E3  &686 × 991   \\\cline{1-3}
\end{longtabu}


\paragraph*{Envelopes}

There are three main types of envelopes used in the US and they do not correspond to the I\+SO standard of paper sizes. They are known as Commercial, Announcement and Catalog. Other less known styles of envelope are Baronial, Booklet and Square. To identify the difference between a I\+SO size and the US size, a hyphen is often inserted between the letter and number. For example, A2 becomes A-\/2.

Commercial envelopes, also referred to as Office envelopes, are most commonly used in business situations as they are suitable for automated franking and filling. They are long and thin, with an aspect ration of between 1\+:1.\+6 and 1\+:2.\+2. The most well known commercial envelope is No.\+10, as it is able to fit Letter size paper folded three times, or Legal size paper folded four times parallel to the shorter side.

Announcement envelopes, also known as A series envelopes, are used for greeting cards, invitations and photographs. Their aspect ratio ranges between 1\+:1.\+3 to 1\+:1.\+6, which makes them more of a square shape, rather than long and thin like the Commercial envelopes. The most popular announcement envelopes are Lady Grey (A2) and Diplomat (A9).

Catalog envelopes are most commonly used for catalogs, brochures, and are made with a center seam to make them more durable.\+Their aspect ratio ranges between 1\+:1.\+3 and 1\+:1.\+5, making them very similar shape to the announcement envelopes.

\tabulinesep=1mm
\begin{longtabu} spread 0pt [c]{*{3}{|X[-1]}|}
\hline
\rowcolor{\tableheadbgcolor}\textbf{ enum  }&\multicolumn{2}{p{(\linewidth-\tabcolsep*3-\arrayrulewidth*2)*2/3}|}{\cellcolor{\tableheadbgcolor}\textbf{ Size Nam   }}\\\cline{1-3}
\endfirsthead
\hline
\endfoot
\hline
\rowcolor{\tableheadbgcolor}\textbf{ enum  }&\multicolumn{2}{p{(\linewidth-\tabcolsep*3-\arrayrulewidth*2)*2/3}|}{\cellcolor{\tableheadbgcolor}\textbf{ Size Nam   }}\\\cline{1-3}
\endhead
U\+S\+\_\+\+Envelope\+\_\+6\+\_\+14  &6¼  &152 × 89   \\\cline{1-3}
U\+S\+\_\+\+Envelope\+\_\+6\+\_\+34  &6¾  &165 × 92   \\\cline{1-3}
U\+S\+\_\+\+Envelope\+\_\+7  &7  &172 × 95   \\\cline{1-3}
U\+S\+\_\+\+Envelope\+\_\+7\+\_\+34\+\_\+\+Monarch  &7¾ Monarch  &191 × 98   \\\cline{1-3}
U\+S\+\_\+\+Envelope\+\_\+8\+\_\+58  &8⅝  &219 × 92   \\\cline{1-3}
U\+S\+\_\+\+Envelope\+\_\+9  &9  &225 × 98   \\\cline{1-3}
U\+S\+\_\+\+Envelope\+\_\+10  &10  &241 × 104   \\\cline{1-3}
U\+S\+\_\+\+Envelope\+\_\+11  &11  &264 × 114   \\\cline{1-3}
U\+S\+\_\+\+Envelope\+\_\+12  &12  &279 × 12   \\\cline{1-3}
U\+S\+\_\+\+Envelope\+\_\+14  &14  &292 × 127   \\\cline{1-3}
U\+S\+\_\+\+Envelope\+\_\+16  &16  &305 × 152   \\\cline{1-3}
U\+S\+\_\+\+Envelope\+\_\+\+A1  &A1  &92 × 130   \\\cline{1-3}
U\+S\+\_\+\+Envelope\+\_\+\+A2\+\_\+\+Lady\+\_\+\+Grey  &A2 Lady Grey  &146 × 111   \\\cline{1-3}
U\+S\+\_\+\+Envelope\+\_\+\+A4  &A4  &159 × 108   \\\cline{1-3}
U\+S\+\_\+\+Envelope\+\_\+\+A6\+\_\+\+Thompsons\+\_\+\+Standard  &A6 Thompson\textquotesingle{}s Standard  &165 × 121   \\\cline{1-3}
U\+S\+\_\+\+Envelope\+\_\+\+A7\+\_\+\+Besselheim  &A7 Besselheim  &184 × 133   \\\cline{1-3}
U\+S\+\_\+\+Envelope\+\_\+\+A8\+\_\+\+Carrs  &A8 Carr\textquotesingle{}s  &206 × 140   \\\cline{1-3}
U\+S\+\_\+\+Envelope\+\_\+\+A9\+\_\+\+Diplomat  &A9 Diplomat  &222 × 146   \\\cline{1-3}
U\+S\+\_\+\+Envelope\+\_\+\+A10\+\_\+\+Willow  &A10 Willow  &241 × 152   \\\cline{1-3}
U\+S\+\_\+\+Envelope\+\_\+\+A\+\_\+\+Long  &A Long  &225 × 98   \\\cline{1-3}
U\+S\+\_\+\+Envelope\+\_\+1  &1  &229 × 152   \\\cline{1-3}
U\+S\+\_\+\+Envelope\+\_\+1\+\_\+34  &1¾  &241 × 152   \\\cline{1-3}
U\+S\+\_\+\+Envelope\+\_\+3  &3  &254 × 178   \\\cline{1-3}
U\+S\+\_\+\+Envelope\+\_\+6  &6  &267 × 191   \\\cline{1-3}
U\+S\+\_\+\+Envelope\+\_\+8  &8  &286 × 210   \\\cline{1-3}
U\+S\+\_\+\+Envelope\+\_\+9\+\_\+34  &9¾  &286 × 222   \\\cline{1-3}
U\+S\+\_\+\+Envelope\+\_\+10\+\_\+12  &10½  &305 × 229   \\\cline{1-3}
U\+S\+\_\+\+Envelope\+\_\+12\+\_\+12  &12½  &318 × 241   \\\cline{1-3}
U\+S\+\_\+\+Envelope\+\_\+13\+\_\+12  &13½  &330 × 254   \\\cline{1-3}
U\+S\+\_\+\+Envelope\+\_\+14\+\_\+12  &14½  &368 × 292   \\\cline{1-3}
U\+S\+\_\+\+Envelope\+\_\+15  &15  &381 × 254   \\\cline{1-3}
U\+S\+\_\+\+Envelope\+\_\+15\+\_\+12  &15½  &394 × 305   \\\cline{1-3}
\end{longtabu}


\subsubsection*{Japanese Sizes}

\paragraph*{J\+IS B Sizes}

The Japanese Industrial Standards (J\+IS), defines two main series of paper sizes. They are the J\+IS A Series, and the J\+IS B Series. Both of the these series are widely available in Japan, as well as China and Taiwan. The J\+IS A Series is exactly the same as the I\+SO A Series, but with different tolerances. Both the J\+IS A Series and the J\+IS B Series have the same aspect ratio, but the area of the B Series paper is 1.\+5 times larger than the A Series. As well as the J\+IS A and B Series, there are a number of traditional Japanese paper sizes used mostly by printers. These include Shiroku-\/ban and Kiku, which are the most commonly used.

\tabulinesep=1mm
\begin{longtabu} spread 0pt [c]{*{3}{|X[-1]}|}
\hline
\rowcolor{\tableheadbgcolor}\textbf{ enum  }&\multicolumn{2}{p{(\linewidth-\tabcolsep*3-\arrayrulewidth*2)*2/3}|}{\cellcolor{\tableheadbgcolor}\textbf{ Size Nam   }}\\\cline{1-3}
\endfirsthead
\hline
\endfoot
\hline
\rowcolor{\tableheadbgcolor}\textbf{ enum  }&\multicolumn{2}{p{(\linewidth-\tabcolsep*3-\arrayrulewidth*2)*2/3}|}{\cellcolor{\tableheadbgcolor}\textbf{ Size Nam   }}\\\cline{1-3}
\endhead
J\+I\+S\+\_\+\+B0  &B0  &1030 × 1456   \\\cline{1-3}
J\+I\+S\+\_\+\+B1  &B1  &728 × 1030   \\\cline{1-3}
J\+I\+S\+\_\+\+B2  &B2  &515 × 728   \\\cline{1-3}
J\+I\+S\+\_\+\+B3  &B3  &364 × 515   \\\cline{1-3}
J\+I\+S\+\_\+\+B4  &B4  &257 × 364   \\\cline{1-3}
J\+I\+S\+\_\+\+B5  &B5  &182 × 257   \\\cline{1-3}
J\+I\+S\+\_\+\+B6  &B6  &128 × 182   \\\cline{1-3}
J\+I\+S\+\_\+\+B7  &B7  &91 × 128   \\\cline{1-3}
J\+I\+S\+\_\+\+B8  &B8  &64 × 91   \\\cline{1-3}
J\+I\+S\+\_\+\+B9  &B9  &45 × 64   \\\cline{1-3}
J\+I\+S\+\_\+\+B10  &B10  &32 × 45   \\\cline{1-3}
J\+I\+S\+\_\+\+B11  &B11  &22 × 32   \\\cline{1-3}
J\+I\+S\+\_\+\+B12  &B12  &16 × 22   \\\cline{1-3}
\end{longtabu}


\paragraph*{Non-\/standard and Traditional paper sizes.}

\tabulinesep=1mm
\begin{longtabu} spread 0pt [c]{*{3}{|X[-1]}|}
\hline
\rowcolor{\tableheadbgcolor}\textbf{ num  }&\multicolumn{2}{p{(\linewidth-\tabcolsep*3-\arrayrulewidth*2)*2/3}|}{\cellcolor{\tableheadbgcolor}\textbf{ Size Name   }}\\\cline{1-3}
\endfirsthead
\hline
\endfoot
\hline
\rowcolor{\tableheadbgcolor}\textbf{ num  }&\multicolumn{2}{p{(\linewidth-\tabcolsep*3-\arrayrulewidth*2)*2/3}|}{\cellcolor{\tableheadbgcolor}\textbf{ Size Name   }}\\\cline{1-3}
\endhead
J\+I\+S\+\_\+\+A\+\_\+ban  &A-\/ban  &625 x 880   \\\cline{1-3}
J\+I\+S\+\_\+\+B\+\_\+ban  &B-\/ban  &765 x 1085   \\\cline{1-3}
Japanese\+\_\+\+AB  &AB  &210 x 257   \\\cline{1-3}
Japanese\+\_\+\+B40  &B40  &103 x 182   \\\cline{1-3}
Japanese\+\_\+\+Shikisen  &Shikisen  &84 x 148   \\\cline{1-3}
Japanese\+\_\+\+A\+\_\+\+Koban  &A-\/\+Koban  &608 x 866   \\\cline{1-3}
Japanese\+\_\+\+B\+\_\+\+Koban  &B-\/\+Koban  &754 x 1047   \\\cline{1-3}
Japanese\+\_\+\+Shirokuban\+\_\+4  &Shiroku ban 4  &264 × 379   \\\cline{1-3}
Japanese\+\_\+\+Shirokuban\+\_\+5  &Shiroku ban 5  &189 × 26   \\\cline{1-3}
Japanese\+\_\+\+Shirokuban\+\_\+6  &Shiroku ban 6  &127 × 188   \\\cline{1-3}
Japanese\+\_\+\+Kiku\+\_\+4  &Kiku 4  &227 × 306   \\\cline{1-3}
Japanese\+\_\+\+Kiku\+\_\+5  &Kiku 5  &151 × 22   \\\cline{1-3}
Japanese\+\_\+\+Zenshi  &Zenshi  &457 x 560   \\\cline{1-3}
Japanese\+\_\+\+Han\+\_\+\+Kiri  &Han Kiri  &356 x 432   \\\cline{1-3}
Japanese\+\_\+\+Yatsu\+\_\+giri  &Yatsu-\/giri  &254 x 305   \\\cline{1-3}
Japanese\+\_\+\+Matsu\+\_\+giri  &Matsu-\/giri  &203 x 254   \\\cline{1-3}
Japanese\+\_\+\+Yotsu\+\_\+giri  &Yotsu-\/giri  &165 x 216   \\\cline{1-3}
Japanese\+\_\+\+Dai\+\_\+\+Kyabine  &Dai Kyabine  &130 x 180   \\\cline{1-3}
Japanese\+\_\+\+Kyabine  &Kyabine  &120 x 165   \\\cline{1-3}
Japanese\+\_\+\+Nimai\+\_\+gake  &Nimai-\/ake  &102 x 127   \\\cline{1-3}
Japanese\+\_\+\+Potsutokaado  &Potsutokaado  &89 x 140   \\\cline{1-3}
Japanese\+\_\+\+Dai\+\_\+\+Tefuda  &Dai efuda  &90 x 130   \\\cline{1-3}
Japanese\+\_\+\+Saabisu  &Saabisu  &82 x 114   \\\cline{1-3}
Japanese\+\_\+\+Tefuda  &Tefuda  &76 x 112   \\\cline{1-3}
Japanese\+\_\+\+Dai\+\_\+meishi  &Dai meishi  &65 x 90   \\\cline{1-3}
Japanese\+\_\+\+Kokusai\+\_\+ban  &Kokusai-\/ban  &216 x 280   \\\cline{1-3}
Japanese\+\_\+\+Hyoujun\+\_\+gata  &Hyoujun-\/gata  &177 x 250   \\\cline{1-3}
Japanese\+\_\+\+Oo\+\_\+gata  &Oo-\/gata  &177 x 230   \\\cline{1-3}
Japanese\+\_\+\+Chuu\+\_\+gata  &Chuu-\/gata  &162 x 210   \\\cline{1-3}
Japanese\+\_\+\+Ko\+\_\+gata  &Ko-\/gata  &148 x 210   \\\cline{1-3}
Japanese\+\_\+\+Ippitsu\+\_\+sen  &Ippitsu-\/sen  &82 x 185   \\\cline{1-3}
Japanese\+\_\+\+Hanshi  &Hanshi  &242 x 343   \\\cline{1-3}
Japanese\+\_\+\+Mino  &Mino  &273 x 394   \\\cline{1-3}
Japanese\+\_\+\+Oohousho  &Oohousho  &394 x 530   \\\cline{1-3}
Japanese\+\_\+\+Chuuhousho  &Chuuhousho  &364 x 500   \\\cline{1-3}
Japanese\+\_\+\+Kohousho  &Kohousho  &333 x 470   \\\cline{1-3}
Japanese\+\_\+\+Nisho\+\_\+no\+\_\+uchi  &Nisho no uchi  &333 x 485   \\\cline{1-3}
Japanese\+\_\+\+Kusuma  &Kusuma  &939 x 1757   \\\cline{1-3}
\end{longtabu}


\paragraph*{Envelopes}

\tabulinesep=1mm
\begin{longtabu} spread 0pt [c]{*{3}{|X[-1]}|}
\hline
\rowcolor{\tableheadbgcolor}\textbf{ num  }&\multicolumn{2}{p{(\linewidth-\tabcolsep*3-\arrayrulewidth*2)*2/3}|}{\cellcolor{\tableheadbgcolor}\textbf{ Size Name   }}\\\cline{1-3}
\endfirsthead
\hline
\endfoot
\hline
\rowcolor{\tableheadbgcolor}\textbf{ num  }&\multicolumn{2}{p{(\linewidth-\tabcolsep*3-\arrayrulewidth*2)*2/3}|}{\cellcolor{\tableheadbgcolor}\textbf{ Size Name   }}\\\cline{1-3}
\endhead
Chou 1  &Chou 1  &142 x 332   \\\cline{1-3}
J\+IS Chou 2  &Chou 2  &119 x 277   \\\cline{1-3}
J\+IS Chou 3  &Chou 3  &120 x 235   \\\cline{1-3}
Chou 31  &Chou 31  &105 x 235   \\\cline{1-3}
Chou 30  &Chou 30  &92 x 235   \\\cline{1-3}
Chou 40  &Chou 40  &90 x 225   \\\cline{1-3}
J\+IS Chou 4  &Chou 4  &90 x 205   \\\cline{1-3}
Kaku A3  &Kaku A3  &320 x 440   \\\cline{1-3}
Kaku 0  &Kaku 0  &287 x 382   \\\cline{1-3}
Kaku 1  &Kaku 1  &270 x 382   \\\cline{1-3}
J\+IS Kaku 2  &Kaku 2  &240 x 332   \\\cline{1-3}
Kaku Koku-\/sai A4  &Kaku Koku-\/sai A4  &229 x 324   \\\cline{1-3}
J\+IS Kaku 3  &Kaku 3  &216 x 277   \\\cline{1-3}
J\+IS Kaku 4  &Kaku 4  &197 x 267   \\\cline{1-3}
J\+IS Kaku 5  &Kaku 5  &190 x 240   \\\cline{1-3}
J\+IS Kaku 6  &Kaku 6  &162 x 229   \\\cline{1-3}
J\+IS Kaku 7  &Kaku 7  &142 x 205   \\\cline{1-3}
J\+IS Kaku 8  &Kaku 8  &119 x 197   \\\cline{1-3}
You 0  &You 0 (Non-\/\+J\+IS)  &197 x 136   \\\cline{1-3}
J\+IS You 0  &You 0  &235 x 120   \\\cline{1-3}
J\+IS You 1  &You 1  &173 x 118 $\ast$   \\\cline{1-3}
J\+IS You 2  &You 2  &162 x 114   \\\cline{1-3}
J\+IS You 3  &You 3  &148 x 98   \\\cline{1-3}
J\+IS You 4  &You 4  &235 x 105   \\\cline{1-3}
J\+IS You 5  &You 5  &217 x 95   \\\cline{1-3}
J\+IS You 6  &You 6  &190 x 98   \\\cline{1-3}
J\+IS You 7  &You 7  &165 x 92   \\\cline{1-3}
\end{longtabu}


$\ast$\+There are two envelope sizes named You 1, 173mm x 118mm and 176mm x 120mm. The smaller size however is apparently the industry standard so that is the only one I have included.

\subsubsection*{Chinese Sizes}

\tabulinesep=1mm
\begin{longtabu} spread 0pt [c]{*{3}{|X[-1]}|}
\hline
\rowcolor{\tableheadbgcolor}\textbf{ enum  }&\multicolumn{2}{p{(\linewidth-\tabcolsep*3-\arrayrulewidth*2)*2/3}|}{\cellcolor{\tableheadbgcolor}\textbf{ Size Nam   }}\\\cline{1-3}
\endfirsthead
\hline
\endfoot
\hline
\rowcolor{\tableheadbgcolor}\textbf{ enum  }&\multicolumn{2}{p{(\linewidth-\tabcolsep*3-\arrayrulewidth*2)*2/3}|}{\cellcolor{\tableheadbgcolor}\textbf{ Size Nam   }}\\\cline{1-3}
\endhead
Chinese\+\_\+\+D0  &D0  &764 x 1064   \\\cline{1-3}
Chinese\+\_\+\+D1  &D1  &532 x 760   \\\cline{1-3}
Chinese\+\_\+\+D2  &D2  &380 x 528   \\\cline{1-3}
Chinese\+\_\+\+D3  &D3  &264 x 376   \\\cline{1-3}
Chinese\+\_\+\+D4  &D4  &188 x 260   \\\cline{1-3}
Chinese\+\_\+\+D5  &D5  &130 x 184   \\\cline{1-3}
Chinese\+\_\+\+D6  &D6  &92 x 126   \\\cline{1-3}
Chinese\+\_\+\+R\+D0  &R\+D0  &787 x 1092   \\\cline{1-3}
Chinese\+\_\+\+R\+D1  &R\+D1  &546 x 787   \\\cline{1-3}
Chinese\+\_\+\+R\+D2  &R\+D2  &393 x 546   \\\cline{1-3}
Chinese\+\_\+\+R\+D3  &R\+D3  &273 x 393   \\\cline{1-3}
Chinese\+\_\+\+R\+D4  &R\+D4  &196 x 273   \\\cline{1-3}
Chinese\+\_\+\+R\+D5  &R\+D5  &136 x 196   \\\cline{1-3}
Chinese\+\_\+\+R\+D6  &R\+D6  &98 x 136   \\\cline{1-3}
\end{longtabu}


\subsubsection*{Canadian Sizes}

\tabulinesep=1mm
\begin{longtabu} spread 0pt [c]{*{3}{|X[-1]}|}
\hline
\rowcolor{\tableheadbgcolor}\textbf{ enum  }&\multicolumn{2}{p{(\linewidth-\tabcolsep*3-\arrayrulewidth*2)*2/3}|}{\cellcolor{\tableheadbgcolor}\textbf{ Size Nam   }}\\\cline{1-3}
\endfirsthead
\hline
\endfoot
\hline
\rowcolor{\tableheadbgcolor}\textbf{ enum  }&\multicolumn{2}{p{(\linewidth-\tabcolsep*3-\arrayrulewidth*2)*2/3}|}{\cellcolor{\tableheadbgcolor}\textbf{ Size Nam   }}\\\cline{1-3}
\endhead
Canadian\+\_\+\+P1  &P1  &560 × 860   \\\cline{1-3}
Canadian\+\_\+\+P2  &P2  &430 × 560   \\\cline{1-3}
Canadian\+\_\+\+P3  &P3  &280 × 430   \\\cline{1-3}
Canadian\+\_\+\+P4  &P4  &215 × 280   \\\cline{1-3}
Canadian\+\_\+\+P5  &P5  &140 × 215   \\\cline{1-3}
Canadian\+\_\+\+P6  &P6  &107 × 140   \\\cline{1-3}
\end{longtabu}


\subsubsection*{Columbian Sizes}

\tabulinesep=1mm
\begin{longtabu} spread 0pt [c]{*{3}{|X[-1]}|}
\hline
\rowcolor{\tableheadbgcolor}\textbf{ enum  }&\multicolumn{2}{p{(\linewidth-\tabcolsep*3-\arrayrulewidth*2)*2/3}|}{\cellcolor{\tableheadbgcolor}\textbf{ Size Nam   }}\\\cline{1-3}
\endfirsthead
\hline
\endfoot
\hline
\rowcolor{\tableheadbgcolor}\textbf{ enum  }&\multicolumn{2}{p{(\linewidth-\tabcolsep*3-\arrayrulewidth*2)*2/3}|}{\cellcolor{\tableheadbgcolor}\textbf{ Size Nam   }}\\\cline{1-3}
\endhead
Columbian\+\_\+\+Carta  &Carta  &216 × 279   \\\cline{1-3}
Columbian\+\_\+\+Extra\+\_\+\+Tabloide  &Extra Tabloide  &304 × 457$\ast$   \\\cline{1-3}
Columbian\+\_\+\+Oficio  &Oficio  &216 × 330   \\\cline{1-3}
Columbian\+\_\+1\+\_\+8\+\_\+\+Pliego  &⅛ Pliego  &250 × 350   \\\cline{1-3}
Columbian\+\_\+1\+\_\+4\+\_\+\+Pliego  &¼ Pliego  &350 × 500   \\\cline{1-3}
Columbian\+\_\+1\+\_\+2\+\_\+\+Pliego  &½ Pliego  &500 × 700   \\\cline{1-3}
Columbian\+\_\+\+Pliego  &Pliego  &700 × 1000   \\\cline{1-3}
\end{longtabu}



\begin{DoxyItemize}
\item Actual size is 304mm × 457.\+2mm but as I am only returning integers the returned size is only 457mm.
\end{DoxyItemize}

\subsubsection*{French Sizes}

\tabulinesep=1mm
\begin{longtabu} spread 0pt [c]{*{3}{|X[-1]}|}
\hline
\rowcolor{\tableheadbgcolor}\textbf{ enum  }&\multicolumn{2}{p{(\linewidth-\tabcolsep*3-\arrayrulewidth*2)*2/3}|}{\cellcolor{\tableheadbgcolor}\textbf{ Size Nam   }}\\\cline{1-3}
\endfirsthead
\hline
\endfoot
\hline
\rowcolor{\tableheadbgcolor}\textbf{ enum  }&\multicolumn{2}{p{(\linewidth-\tabcolsep*3-\arrayrulewidth*2)*2/3}|}{\cellcolor{\tableheadbgcolor}\textbf{ Size Nam   }}\\\cline{1-3}
\endhead
French\+\_\+\+Cloche  &Cloche  &300 × 400   \\\cline{1-3}
French\+\_\+\+Pot\+\_\+ecolier  &Pot, écolier  &310 × 400   \\\cline{1-3}
French\+\_\+\+Telliere  &Tellière  &340 × 440   \\\cline{1-3}
French\+\_\+\+Couronne\+\_\+ecriture  &Couronne écriture  &360 × 360   \\\cline{1-3}
French\+\_\+\+Couronne\+\_\+edition  &Couronne édition  &370 × 470   \\\cline{1-3}
French\+\_\+\+Roberto  &Roberto  &390 × 500   \\\cline{1-3}
French\+\_\+\+Ecu  &Écu  &400 × 520   \\\cline{1-3}
French\+\_\+\+Coquille  &Coquille  &440 × 560   \\\cline{1-3}
French\+\_\+\+Carre  &Carré  &450 × 560   \\\cline{1-3}
French\+\_\+\+Cavalier  &Cavalier  &460 × 620   \\\cline{1-3}
French\+\_\+\+Demi\+\_\+raisin  &Demi-\/raisin  &325 × 500   \\\cline{1-3}
French\+\_\+\+Raisin  &Raisin  &500 × 650   \\\cline{1-3}
French\+\_\+\+Double\+\_\+\+Raisin  &Double Raisin  &650 × 1000   \\\cline{1-3}
French\+\_\+\+Jesus  &Jésus  &560 × 760   \\\cline{1-3}
French\+\_\+\+Soliel  &Soleil  &600 × 800   \\\cline{1-3}
French\+\_\+\+Colombier\+\_\+\+Affiche  &Colombier affiche  &600 × 800   \\\cline{1-3}
French\+\_\+\+Colombier\+\_\+\+Commercial  &Colombier commercial  &630 × 900   \\\cline{1-3}
French\+\_\+\+Petit\+\_\+\+Aigle  &Petit Aigle  &700 × 940   \\\cline{1-3}
French\+\_\+\+Grand\+\_\+\+Aigle  &Grand Aigle  &750 × 1050   \\\cline{1-3}
French\+\_\+\+Grand\+\_\+\+Monde  &Grand Monde  &900 × 1260   \\\cline{1-3}
French\+\_\+\+Univers  &Univers  &1000 × 1130   \\\cline{1-3}
\end{longtabu}


\subsubsection*{British Traditional and Transitional Sizes.}

When the United Kingdom started using I\+SO sizes, the traditional paper sizes were no longer commonly used. Most of the traditional sizes were used in the production of books, and were never used for other stationery purposes. ‘\+Folio’ or ‘\+Foolscap’ is an alias for Foolscap Folio, as is ‘\+Kings’ being an alias for ‘\+Foolscap Quarto’.

\tabulinesep=1mm
\begin{longtabu} spread 0pt [c]{*{3}{|X[-1]}|}
\hline
\rowcolor{\tableheadbgcolor}\textbf{ enum  }&\multicolumn{2}{p{(\linewidth-\tabcolsep*3-\arrayrulewidth*2)*2/3}|}{\cellcolor{\tableheadbgcolor}\textbf{ Size Nam   }}\\\cline{1-3}
\endfirsthead
\hline
\endfoot
\hline
\rowcolor{\tableheadbgcolor}\textbf{ enum  }&\multicolumn{2}{p{(\linewidth-\tabcolsep*3-\arrayrulewidth*2)*2/3}|}{\cellcolor{\tableheadbgcolor}\textbf{ Size Nam   }}\\\cline{1-3}
\endhead
British\+\_\+\+Dukes  &Dukes  &140 × 178   \\\cline{1-3}
British\+\_\+\+Foolscap  &Foolscap  &203 × 330   \\\cline{1-3}
British\+\_\+\+Imperial  &Imperial  &178 × 229   \\\cline{1-3}
British\+\_\+\+Kings  &Kings  &165 × 203   \\\cline{1-3}
British\+\_\+\+Quarto  &Quarto  &203 × 254   \\\cline{1-3}
\end{longtabu}


Before the United Kingdom adopted the I\+SO 216 standard, British Imperial paper sizes were used. The Imperial paper sizes were used to define large sheets of paper, and the naming convention was derived from the sheet name, and how many times it was folded. For example ‘\+Royal’ paper is 508mm x 635mm, and when it is folded three times, it makes 8 sheets and goes by the name ‘\+Royal Octavo’ which is 253 mm x 158 mm. This is a name given to modern hardbound books.

\tabulinesep=1mm
\begin{longtabu} spread 0pt [c]{*{3}{|X[-1]}|}
\hline
\rowcolor{\tableheadbgcolor}\textbf{ enum  }&\multicolumn{2}{p{(\linewidth-\tabcolsep*3-\arrayrulewidth*2)*2/3}|}{\cellcolor{\tableheadbgcolor}\textbf{ Size Nam   }}\\\cline{1-3}
\endfirsthead
\hline
\endfoot
\hline
\rowcolor{\tableheadbgcolor}\textbf{ enum  }&\multicolumn{2}{p{(\linewidth-\tabcolsep*3-\arrayrulewidth*2)*2/3}|}{\cellcolor{\tableheadbgcolor}\textbf{ Size Nam   }}\\\cline{1-3}
\endhead
Imperial\+\_\+\+Antiquarian  &Antiquarian  &787 × 1346   \\\cline{1-3}
Imperial\+\_\+\+Atlas  &Atlas  &660 × 864   \\\cline{1-3}
Imperial\+\_\+\+Brief  &Brief  &343 × 406   \\\cline{1-3}
Imperial\+\_\+\+Broadsheet  &Broadsheet  &457 × 610   \\\cline{1-3}
Imperial\+\_\+\+Cartridge  &Cartridge  &533 × 660   \\\cline{1-3}
Imperial\+\_\+\+Columbier  &Columbier  &597 × 876   \\\cline{1-3}
Imperial\+\_\+\+Copy\+\_\+\+Draught  &Copy Draught  &406 × 508   \\\cline{1-3}
Imperial\+\_\+\+Crown  &Crown  &381 × 508   \\\cline{1-3}
Imperial\+\_\+\+Demy  &Demy  &445 × 572   \\\cline{1-3}
Imperial\+\_\+\+Double\+\_\+\+Demy  &Double Demy  &572 × 902   \\\cline{1-3}
Imperial\+\_\+\+Quad\+\_\+\+Demy  &Quad Demy  &889 × 1143   \\\cline{1-3}
Imperial\+\_\+\+Elephant  &Elephant  &584 × 711   \\\cline{1-3}
Imperial\+\_\+\+Double\+\_\+\+Elephant  &Double Elephant  &678 × 1016   \\\cline{1-3}
Imperial\+\_\+\+Emperor  &Emperor  &1219 × 1829   \\\cline{1-3}
Imperial\+\_\+\+Foolscap  &Foolscap  &343 × 432   \\\cline{1-3}
Imperial\+\_\+\+Small\+\_\+\+Foolscap  &Small Foolscap  &337 × 419   \\\cline{1-3}
Imperial\+\_\+\+Grand\+\_\+\+Eagle  &Grand Eagle  &730 × 1067   \\\cline{1-3}
Imperial\+\_\+\+Imperial  &Imperial  &559 × 762   \\\cline{1-3}
Imperial\+\_\+\+Medium  &Medium  &470 × 584   \\\cline{1-3}
Imperial\+\_\+\+Monarch  &Monarch  &184 × 267   \\\cline{1-3}
Imperial\+\_\+\+Post  &Post  &394 × 489   \\\cline{1-3}
Imperial\+\_\+\+Sheet\+\_\+\+Half\+\_\+\+Post  &Sheet, Half Post  &495 × 597   \\\cline{1-3}
Imperial\+\_\+\+Pinched\+\_\+\+Post  &Pinched Post  &375 × 470   \\\cline{1-3}
Imperial\+\_\+\+Large\+\_\+\+Post  &Large Post  &394 × 508   \\\cline{1-3}
Imperial\+\_\+\+Double\+\_\+\+Large\+\_\+\+Post  &Double Large Post  &533 × 838   \\\cline{1-3}
Imperial\+\_\+\+Double\+\_\+\+Post  &Double Post  &483 × 762   \\\cline{1-3}
Imperial\+\_\+\+Pott  &Pott  &318 × 381   \\\cline{1-3}
Imperial\+\_\+\+Princess  &Princess  &546 × 711   \\\cline{1-3}
Imperial\+\_\+\+Quarto  &Quarto  &229 × 279   \\\cline{1-3}
Imperial\+\_\+\+Royal  &Royal  &508 × 635   \\\cline{1-3}
Imperial\+\_\+\+Super\+\_\+\+Royal  &Super Royal  &483 × 686   \\\cline{1-3}
\end{longtabu}
